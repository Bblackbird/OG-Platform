\section{Swap Specifications}

For the treatment of swaps within analytics we define the following quantities:

\begin{itemize}
\item $t_1^{fix} \ldots t_M^{fix}$ is the set of fixed payment times 
\item $t_1^{flt} \ldots t_N^{flt}$ is the set of floating payment times
\item  $F_i$ is the reference rate for the floating payment at $t_i^{flt}$
\item  $\alpha^{fix}_i = \alpha^{fix}(t_{i-1}^{fix},t_{i}^{fix})$ is the period (accrual fraction), according to the day count convention, between fixed payments $i-1$ and $i$
\item  $\alpha^{flt}_i = \alpha^{flt}(t_{i-1}^{flt},t_{i}^{flt})$ is the period (accrual fraction), according to the day count convention, between floating payments $i-1$ and $i$
\item  $\alpha^{F}_i$ is the accrual fraction for reference rate  $F_i$
\item  $\Delta^s_i$ is the difference between the start (effective or fixing) time of $F_i$ and $t^{flt}_{i-1}$
\item $\Delta^e_i$ is the difference between the end (maturity) time of $F_i$ and $t^{flt}_{i}$
\item $p^{fund}(T)$ is the funding discount curve
\end{itemize}

All times are expressed in years (i.e 0.5 is half a year). $t_0^{flt}$ is taken as the effective (start) date of the swap (not a payment date), which is usually two days after the trade date.  In general, one would expect  the start date of some reference rate to be the same as the end date of the previous reference rate, i.e.  $\Delta^s_{i+1}=\Delta^e_{i}$ which are typically of order a few days. 

The present value of the fixed leg is

\begin{equation}
PV(\text{fixed}) = \sum^M_{i=1} k \alpha^{fix}_ip^{fund}(t^{fix}_i)
\end{equation}

where $k$ is the fixed rate. The PV of the floating leg is 

\begin{equation}
PV(\text{float}) = \sum^N_{i=1} F_i \alpha^{flt}_ip^{fund}(t^{flt}_i)
\end{equation}

The PV of a payer swap (i.e. pay the fixed leg) is
\begin{equation}
\begin{split}
PV(\text{payer swap}) &= PV(\text{float}) - PV(\text{fixed}) \\
&=  \sum^N_{i=1} F_i \alpha^{flt}_ip^{fund}(t^{flt}_i)-\sum^M_{i=1} k \alpha^{fix}_ip^{fund}(t^{fix}_i)
\end{split}
\end{equation}

A par swap has a swap rate such that its PV is zero. With this condition, the par-swap rate is
\begin{equation}
\label{eqn:SwapRate}
k = \frac{\sum^N_{i=1} F_i \alpha^{flt}_ip^{fund}(t^{flt}_i)}{ \sum^M_{i=1} \alpha^{fix}_ip^{fund}(t^{fix}_i)}
\end{equation}

\subsection{Libor Reference Rate}

We consider the (usual) case where the reference rate is Libor (with some tenor).  To keep things completely general, we assume that funding is not at Libor, and we define  $p^{libor}(T)$ as the Libor discount curve. The (forward) Libor reference rates can be expressed in terms of the discount factors as

\begin{equation}
\label{eqn:FowardRate}
F_i = \frac{1}{\alpha^{F}_i }\left(\frac{p^{libor}(t^{flt}_{i-1}+\Delta^s_i)} {p^{libor}(t^{flt}_i+\Delta^e_i)} -1\right)
\end{equation}

with the expression for the par-swap rate becoming

\begin{equation}
\label{eqn:ParSwapRate}
k = \frac{\sum^N_{i=1}  \frac{\alpha^{flt}_i}{\alpha^{F}_i} \left(\frac{p^{libor}(t^{flt}_{i-1}+\Delta^s_i)} {p^{libor}(t^{flt}_i+\Delta^e_i)} -1\right)p^{fund}(t^{flt}_i)}{ \sum^M_{i=1} \alpha^{fix}_ip^{fund}(t^{fix}_i)}
\end{equation}

Almost certainly we have $\alpha^{flt}_i = \alpha^{F}_i$, but it is left in the expression for maximum generality. In the case that all the $\Delta$s are zero and the funding curve is Libor, equation  \ref{eqn:SwapRate} collapses to the usual textbook form

\begin{equation}
k = \frac{p(t^{flt}_0)-p(t^{flt}_N)}{ \sum^M_{i=1} \alpha^{fix}_ip(t^{fix}_i)}
\end{equation}

In both cases the swap rate is a function of the discount factors (possibly from multiple curves) at fixed times.


\section{Swap Rate Sensitivities}
\label{Sec:SwapRateSense}
Let the fixed leg annuity be $A$ (i.e. $PV(fixed)$ with $k=1$) and $PV(float)=B$, then

\begin{equation}
\label{eqn:SwaprateSense}
\frac{\partial k}{\partial p^x(t)}=\frac{1}{A}\frac{\partial B}{\partial p^x(t)} - \frac{B}{A^2}\frac{\partial A}{\partial p^x(t)} 
\end{equation}

where the superscript $x$ indicates with respect to which curve we are taking the derivative. The fixed leg part is

\begin{equation}
\frac{\partial A}{\partial p^{x}(t)}=
\begin{cases}
\alpha^{fix}_k \mathbf{1}(t^{fix}_k-t)& \text{if $x$ is the funding curve},\\
0 \quad \forall t & \text{if $x$ is any other curve}
\end{cases}
\end{equation}

where $\mathbf{1}(x)$ is the indicator function which equals 1 if  $x=0$ and is 0 otherwise. The most general form of the floating leg is 

\begin{equation}
\label{eqn:FloatSense}
\frac{\partial B}{\partial p^{x}(t)}=
\begin{cases}
\alpha^{flt}_k F_k\mathbf{1}(t^{flt}_k-t)& \text{if $x$ is the funding curve},\\
\sum^N_{i=1} \frac{\partial F_i}{\partial p^{x}(t)} \alpha^{flt}_ip^{fund}(t^{flt}_i) \quad \forall t & \text{if $x$ is any other curve}
\end{cases}
\end{equation}

If the reference rate is linked to the funding curve (e.g. the reference is Libor and funding is at Libor) then both terms in the above equation must be considered. Returning to the case of a Libor reference rate, we have

\begin{equation}
\label{eqn:ForwardRateSense}
\begin{split}
\frac{\partial F_i}{\partial p(t)} &= \frac{1}{\alpha^{F}_i }\frac{\partial}{\partial p(t)} \left[ \frac{p(t^{flt}_{i-1}+\Delta^s_i)} {p(t^{flt}_i+\Delta^e_i)} -1\right]\\
&=\frac{\mathbf{1}(t^{flt}_{i-1}+\Delta^s_i-t)} {p(t^{flt}_i+\Delta^e_i)} - \frac{p(t^{flt}_{i-1}+\Delta^s_i)\mathbf{1}(t^{flt}_{i}+\Delta^e_i-t)} {\left(p(t^{flt}_i+\Delta^e_i)\right)^2}
\end{split}
\end{equation}

with $p(t)$ understood to be the discount factors for the Libor curve.  Putting this together with equation \ref{eqn:FloatSense} we finally have 

\begin{equation}
\label{eqn:FloatLegSense}
\frac{\partial B}{\partial p^{libor}(t)}=
\begin{cases}
 \frac{\alpha^{flt}_k}{\alpha^{F}_k }\frac{p^{fund}(t^{flt}_{k})} {p^{libor}(t^{flt}_k+\Delta^e_k)}& \text{for } t=t^{flt}_{k-1}+\Delta^s_k,\\
 -\frac{\alpha^{flt}_k}{\alpha^{F}_k }p^{fund}(t^{flt}_{k})\frac{p^{libor}(t^{flt}_{k-1}+\Delta^s_k)} {\left(p^{libor}(t^{flt}_k+\Delta^e_k)\right)^2}& \text{for } t=t^{flt}_{k}+\Delta^e_k,\\
 0 & \text{for $t$ taking any other value}
 \end{cases}
\end{equation}

It is clear from equation  \ref{eqn:SwapRate} that the swap rate only depends on the values of discount factors (possibly from multiple curves) at a set of predefined times - the values of the curve away from these times are irrelevant (hence the indicator function popping up in the sensitivities). 

\subsection{Present Value Sensitivity}
Using the same notation as above, the PV of a payer swap is
%
\begin{equation}
PV(\text{payer}) = B-kA
\end{equation}
%
so
%
\begin{equation}
\frac{\partial PV}{\partial p^{x}(t)}= \frac{\partial B}{\partial p^{x}(t)}-k\frac{\partial A}{\partial p^{x}(t)}
\end{equation}
with the sensitivities of the two legs being the same as above. 



\section{Implying Yield Cuves}
A discount curve is a monotonically decreasing function,\footnote{providing the instantaneous forward rates never become negative} so does not provide much visual information nor  the best metric for interpolation. We work with yield curves, with the yield to time $t$ defined as $r^x(t)=-\frac{1}{t}\ln\left(p^x(t)\right)$. Clearly, our sensitivity to a point on yield curve $x$ is
%
\begin{equation}
\frac{\partial \theta}{\partial r^{x}(t)} = -tp^x(t)\frac{\partial \theta}{\partial p^{x}(t)}
\end{equation}
%
where $\theta$ is any curve sensitive quantity (e.g. par swap rate, present value etc). 

Given a yield curve (or set of curves), one can find the swap rates for any number of swaps\footnote{with payments out to the last defined time on the curve}  using equation \ref{eqn:SwapRate}.  We are interested in the inverse problem of finding yield curve(s) given a set of market swap rates for reference par swaps. For some list of swaps, adding an additional swap will add (many) more than one point to the list of required times for which a discount factor is needed. For this reason we must interpolate the yield curve through a set of nodes, and it is these nodes that we aim to find. 

If we have $N$ swaps then we can in principle find a curve with $N$ nodes (or multiple curves with a total of $N$ nodes) that exactly recovers all the swap rates - this is an $N$-dimensional root finding problem.\footnote{for less than $N$ nodes, this can be formulated as a optimisation  problem where one tries to minimise the sum of squares between the market swap rates and those from the candidate curve.}  Alternatively we can find the curve that gives a zero PV to all our swaps - of course this will give exactly the same answer, and has the advantages of allowing more general instruments in the curve construction (not everything has a rate associated with it) and producing a more useful Jacobian (see below).


\subsection{The Jacobian}

The standard method of inverting the vector equation $\mathbf{y} = f(\mathbf{x})$,  when  $\mathbf{x}$ and  $\mathbf{y}$ are the same length, is a Newton-Raphson approach which iterates from some initial guess $\mathbf{x}_0$ using
\begin{equation*}
\mathbf{x}_{i+1} = \mathbf{x}_i - \mathbf{J}^{-1}(f(\mathbf{x})-\mathbf{y} )
\end{equation*}
until $\mathbf{x}_{i+1} - \mathbf{x}_i= \mathbf{0}$ or equivalently $f(\mathbf{x}_i)-\mathbf{y} =\mathbf{0}$. The Jacobian, $\mathbf{J}$ is

\begin{equation*}
\mathbf{J} = \left(  
\begin{array}{cccc}
\frac{\partial y_{1}}{\partial x_1} & \frac{\partial y_1}{\partial x_2} & \ldots & \frac{\partial y_1}{\partial x_N}\\
\frac{\partial y_2}{\partial x_1}&\frac{\partial y_2}{\partial x_2} &\ldots&\frac{\partial y_2}{\partial x_N}\\
\vdots&\ddots&\ddots&\vdots\\
\frac{\partial y_N}{\partial x_1}&\ldots&\ldots&\frac{\partial y_N}{\partial x_N}
\end{array}
\right)
\end{equation*}

Additional heuristics are added to Newton-Raphson to improve stability  and the Broyden method can be used to update the Jacobian rather than recalculate it at every step\footnote{alternatively, one could use Sherman-Morrison to update the inverse Jacobian directly}.

For a set of $N$ swaps, the Jacobian for a single curve is arranged as above, with some instrument value (par swap rates, PV etc) taking the place of the $y$s and the yields taking the place of the $x$s.  For two curves, e.g.  libor and funding, we arrange the Jacobian as follows 
%
\begin{equation}
\mathbf{J} = \left(  
\begin{array}{cccccc}
\frac{\partial \theta_{1}}{\partial r^{libor}_1} & \ldots & \frac{\partial  \theta_1}{\partial r^{libor}_j} &  \frac{\partial  \theta_1}{\partial r^{fund}_1} &\ldots & \frac{\partial  \theta_1}{\partial r^{fund}_{N-j}}\\
\frac{\partial  \theta_{2}}{\partial r^{libor}_1} & \ldots & \frac{\partial s_2}{\partial r^{libor}_j} &  \frac{\partial  \theta_2}{\partial r^{fund}_1} &\ldots & \frac{\partial  \theta_2}{\partial r^{fund}_{N-j}}\\
\vdots & \ddots & \vdots & \vdots & \ddots  &\vdots\\
\frac{\partial  \theta_{N}}{\partial r^{libor}_1} & \ldots & \frac{\partial  \theta_N}{\partial r^{libor}_j} &  \frac{\partial  \theta_N}{\partial r^{fund}_1} &\ldots & \frac{\partial  \theta_N}{\partial r^{fund}_{N-j}}
\end{array}
\right)
\end{equation}
%
with $j$ nodes on the Libor curve and $N-j$ nodes on the funding curve.\footnote{Clearly the vector $\mathbf{x}$ has the $j$ nodes of the Libor curve as its first $j$ entries and the $N-j$ nodes of the funding curves as its remaining entries.} 

When $\theta$ is a PV rather than a rate, the Jacobian provides direct hedging information - the PV01\footnote{Strictly PV01 (present value of a basis point) is the change in PV with respect to a parallel move of the yield curve by one basis point. We have taken the derivative of PV with respect to a parallel shift of the yield curve scaled to one basis point.}  of instrument $i$ is just the sum of terms in the $i^{th}$ row (multiplied by one-thousandth). 

\subsection{Interpolator Sensitivity}

As we have already mentioned, not all of the discount factors needed to calculate the swap rate can be nodes of the yield curve. This means we need to know the sensitivity of any point on the interpolated curve to the movement of the nodes. With this additional information we can form the analytical Jacobian.\footnote{it is possible to obtain the Jacobian using central differencing, which is what we originally did - this requires $2N$ calls to $f(\mathbf{x})$. }


For $N$ nodes $(x_1,y_1), (x_2,y_2),\ldots,(x_N,y_N)$ many interpolators can be written in the form 
%
\begin{equation}
y=\sum^N_{i=1} a_i(x)y_i
\end{equation}
%
The sensitivity of interest is 
\begin{equation}
\frac{\partial y}{\partial y_k}=\sum^N_{i=1}a_i(x)\delta_{i,k} =a_k(x)
\end{equation}

A local interpolator (e.g. linear) will only have sensitivity to the local nodes (i.e. the nodes $k$ and  $k+1$ for $x\in(x_k,x_{k+1})$), while a global method, such as cubic spline, will have sensitivity to all the nodes - this is an important consideration for hedging. 
For an instrument  which is sensitive to a set of yields at times $t_1,\ldots,t_M$, the sensitivity to node $j$ is
%
\begin{equation}
\frac{\partial \theta}{\partial r_j}=\sum_{i=1}^M \frac{\partial r(t_i)}{\partial r_j}\frac{\partial  \theta}{\partial r(t_i)}
\end{equation}

We now have all the pieces in place to imply a yield curve(s) - the inputs are the market instruments (par swaps), the positioning of the yield curve nodes and the choice of interpolator. The output is the yields at those nodes, which (together with the choice of interpolator) will exactly recover the swap rates.  

In order to back out a single curve, the obvious choice of the node positions is at the last payment time of each of the swaps. If a linear interpolator is used, the Jacobian will be lower triangular, and the nodes could be found by bootstrapping - the value of the first node (yield) is chosen to give the correct swap rate for the first swap, with this fixed the second node is chosen to give the correct rate to the second swap, etc. We tested the system with a cubic spline, where every node to some extent has an effect on every swap rate. The system worked well both in an idealised test case\footnote{We constructed a dummy curve, got swap rates off it for USD 6m, 1y, 2y, 3y, 5y, 7y, 10y, 15y, 20y, 25y \& 30y, then used these rates to recover the curve} and using USD reference par-swap rates from Bloomberg.

It is also possible is recover two curves (funding and Libor) from the swap rates. The nodes cannot be placed at the last payment times of the swaps, but instead must be divided up between the two curves. The sensitivity of the swap rates to the funding curve only is small (that part of the Jacobian has much smaller values than the Libor part), and while for an idealised test case both curves can be recovered, there is no guarantee that a root even exists with market data. In practice, additional instruments are used (cash rates, FRAs, futures, basis swaps) to give the required sensitivity to various curves (and particular time periods on those curves). This is dealt with in the next section.



\section{Adding Additional Instruments}

\subsection{Forward Rate Agreements (FRA)}

A FRA has three principle dates associated with it: the fixing date ($t_f$), when the reference Libor rate is observed, the maturity date of that Libor  ($t_m$) - e.g. three months after the fixing date, and the settlement date ($t_s$) when the net payment is made. Unlike swaps, where the payment is at or near maturity, the settlement date of a FRA is close to the fixing date (normally two days after it). 
The actual payment is discounted from a hypothetical payment made at maturity by the observed forward rate.  

\begin{equation}
\text{payment at settlement} = \frac{(L(t_f,t_m)-k)\alpha}{1+\tau L(t_f,t_m)}
\end{equation}
%
where the year fraction to calculate the payment ($\alpha$) and that used to calculate the discount ($\tau$) are not necessarily the same.  The PV of this instrument is given by
\begin{equation}
PV = \mathbb{E}^\mathbb{P}\left[e^{-\int^{t_s}_0 r_t dt} \frac{(L(t_f,t_m)-k)\alpha}{1+\tau L(t_f,t_m)}\right]
\end{equation}
%
where $r_t$ is the funding rate. If funding is at Libor, we can change numeraire to the zero coupon bond maturing at $t_m$, and the expression becomes 
\begin{equation}
\begin{split}
PV &=p^{libor}(t_m) \mathbb{E}^\mathbb{T}\left[(L(t_f,t_m)-k)\alpha\right] \\
&= p^{libor}(t_{f}) -(1+\alpha k) p^{libor}(t_m)
\end{split}
\end{equation}
%
since 
\begin{equation}
\mathbb{E}^\mathbb{T}\left[ L(t_f,t_m)\right] = F(t_f,t_m)= \frac{1}{\alpha}\left(\frac{p^{libor}(t_{f})} {p^{libor}(t_m)} -1\right)
\end{equation}
which is also the strike, $k$, that makes the FRA fair (zero PV).  
 
 In the absence of a model for the joint dynamics of the forward and funding rates, we assume the PV to be given by
 \begin{equation}
 \label{eqn:FRA_PV}
PV = p^{fund}(t_s)\left(\frac{(F(t_f,t_m)-k)\alpha}{1+\tau F(t_f,t_m)}\right)
\end{equation}


For a fair FRA, the sensitivities of the strike purely to the Libor curve given by

\begin{equation}
\frac{\partial k}{\partial p^{libor}(t)}=
\begin{cases}
 \frac{1}{\alpha}\frac{1} {p^{libor}(t_m)}& \text{for } t=t_{f},\\
 - \frac{1}{\alpha }\frac{p^{libor}(t_{f})} {\left(p^{libor}(t_m)\right)^2}& \text{for } t=t_{m},\\
 0 & \text{for $t$ taking any other value}
 \end{cases}
\end{equation}

The PV sensitivity for a Libor-funded FRA is 
\begin{equation}
\frac{\partial PV}{\partial p^{libor}(t)}=
\begin{cases}
 1& \text{for } t=t_{f},\\
  -(1+\alpha k)  & \text{for } t=t_{m},\\
 0 & \text{for $t$ taking any other value}
 \end{cases}
\end{equation}
%
while for some other funding curve it is 
\begin{equation}
\frac{\partial PV}{\partial p^{libor}(t)}=\frac{\alpha p^{fund}(t_s)}{1+\tau F}\left[ 1-\frac{(F-k)\tau}{1+\tau F}  \right] \frac{\partial F}{\partial p^{libor}(t)}
\end{equation}
where $\frac{\partial F}{\partial p(t)}$ is given by equation \ref{eqn:ForwardRateSense}.  Even though the funding curve cannot affect whether an instrument has zero PV (see equation \ref{eqn:FRA_PV}), there is sensitivity to this curve, which is important for calculating the correct hedge ratio. 

\subsection{Futures}
Futures are traded based on price. The price implies a forward  rate given by $F = 1-\frac{\text{price}}{100}$, which is usually 3-month Libor with fixing date the third Wednesday of the contract month. The sensitivity of the implied rate can be treated the same as FRAs above  (i.e. we do not apply any convexity adjustment). 

No money is paid upfront for a future. For a tenor of $\alpha$, if the price rises by $\Delta$ (implying a fall in the forward rate of $\Delta/100$) , there is an immediate profit (paid into a margin account) of $N\Delta \alpha/100$ where $N$ is the notional of one contract.\footnote{For Eurodollar futures  $\alpha$ is 0.25 (i.e 90/360) and the notional is \$1M so a 1bp fall in the forward rate gives a gain of \$25. The minimum change of value (tick) is \$12.50 or 0.5 bps}
So the PV can be written in terms of the current implied forward, $F(t_s,t_m)$, and the price of the future when  the contact was entered, $c$.
%
\begin{equation}
PV = \alpha N\left(1-F(t_s,t_m) -\frac{c}{100}\right)
\end{equation}
%
Then
\begin{equation}
\frac{\partial PV}{\partial p^{libor}(t)}=-\alpha N\frac{\partial F}{\partial p^{libor}(t)}
\end{equation}
%
where again $\frac{\partial F}{\partial p(t)}$ is given by equation \ref{eqn:ForwardRateSense}. So a future has the opposite sign of PV sensitivity to a FRA, and also no sensitivity to the funding curve due to mark-to-marketing. 


\subsection{Spot Libor}
Today's Libor (set at 11am) is available for a range of tenors (1 month, 2 months and 3 months being the most liquid). To be completely consistent when building a Libor curve, we should only use the tenor used in the swaps and FRAs (3 months for USD swaps and FRAs). In the OpenGamma system, spot Libor is treated as a FRA with fixing date and settlement date equal to the trade date, which is normally two days from now.

\subsection{Cash}
Short term cash deposits (i.e. overnight O/N, 1W, 2W etc) which use simple interest can be treated as spot Libor above, but with sensitivity to the funding curve (assuming cash rates represent short term funding costs).  

\subsection{Floating Rate Notes (FRN)}
\label{sec:FRN}
FRNs are bonds that pay a variable coupon based on a reference rate plus a spread. The notional is paid up front and returned at maturity. Using the same notation as for swaps, the present value (with unit notional)  is:
%
\begin{equation}
PV = - p^{fund}(t_0) + p^{fund}(t_N) +  \sum^N_{i=1} (F_i+s) \alpha_ip^{fund}(t_i)
\end{equation}
%
where  $t_0$ is the trade time (which may be zero) and $s$ is the spread. If funding is at Libor, with a zero spread the PV is zero - FRNs are always at par when funding at the reference rate.\footnote{Strictly we  also need $\Delta^s_i$ and  $\Delta^e_i$ to be zero, and $\alpha^{flt}_i=\alpha^F_i$}
To trade at par for any other funding requires a spread of
%
\begin{equation}
s =  \frac{p^{fund}(t_0) - p^{fund}(t_N)- \sum^N_{i=1} F_i \alpha_ip^{fund}(t_i) }{ \sum^N_{i=1} \alpha_ip^{fund}(t_i)}
\end{equation}
%
which will be negative if one can fund at less than Libor (e.g. by posting collateral). The sensitivity of this spread and the PV to the curves will follow through as for swaps. 

\subsection{Basis Swaps}

Basis swaps are float-for-float with each leg based on a different reference rate (interest-rate indices), with one leg typically paying a (fixed) spread.

\paragraph{Basis Swap in a Single Currency}

A basis swap exchanges payments based on one index (e.g.Fed-Funds) for another (e.g. 3m-Libor) on the same notional amount. A tenor swap exchanges payments based on different Libor tenors, e.g. 3 month Libor paid quarterly for  6 month Libor paid semi-annually. 

If legs of the swap ($a$ and $b$) have reference rates $F^a$ and $F^b$, the PV of the receiver of leg $a$ is 
\begin{equation}
PV = \sum^M_{i=1} F^a_i \alpha^{a}_ip^{fund}(t^{a}_i) -  \sum^N_{i=1} (F^b_i+s) \alpha^{b}_ip^{fund}(t^{b}_i)
\end{equation}
 %
where $s$ is the spread, which can take positive or negative values. Making the PV zero, the spread becomes
 
 \begin{equation}
s = \frac{\sum^M_{i=1} F^a_i \alpha^{a}_ip^{fund}(t^{a}_i)-\sum^N_{i=1} F^b_i \alpha^{b}_ip^{fund}(t^{b}_i)}{ \sum^N_{i=1} \alpha^{b}_ip^{fund}(t^{b}_i)}
\end{equation}

If both leg payments are at the same time, this reduces to
%
 \begin{equation}
s = \frac{\sum^M_{i=1} (F^a_i-F^b_i) \alpha_ip^{fund}(t_i)}{ \sum^M_{i=1} \alpha_ip^{fund}(t_i)}
\end{equation}
%
which is just equation \ref{eqn:SwapRate} for the par swap rate, with $F_i = (F^a_i-F^b_i)$.  In the general case,  we again let the spread $s=B/A$ and proceed as in Section \ref{Sec:SwapRateSense}.

\begin{align}
\frac{\partial A}{\partial p^{x}(t)}&=
\begin{cases}
\alpha^{b}_k \mathbf{1}(t^{b}_k-t)& \text{if $x$ is the funding curve},\\
0 \quad \forall t & \text{if $x$ is any other curve}
\end{cases}\\
%
\frac{\partial B}{\partial p^{fund}(t)}&=
\begin{cases}
\alpha^{a}_k F^a_k& \quad t=t^a_k,\\
-\alpha^{b}_k F^b_k& \quad t=t^b_k,\\
0& \text{any other }t
\end{cases}\\
%
\frac{\partial B}{\partial p^{x}(t)}&=\sum^M_{i=1} \frac{\partial F^a_i}{\partial p^{x}(t)} \alpha^{a}_ip^{fund}(t^{a}_i) -\sum^n_{i=1} \frac{\partial F^b_i}{\partial p^{x}(t)} \alpha^{b}_ip^{fund}(t^{b}_i) 
\end{align}
%
If  the reference rates are Libor (with different tenors) then terms like $\frac{\partial F^a_i}{\partial p^{x}(t)}$  are given by equation \ref{eqn:ForwardRateSense}. The PV sensitivity is similarly calculated. 

\paragraph*{Example: Implying two curves - 3m Libor and Fed-Funds}  
A general tenor swap will have sensitivity to three curves.\footnote{funding and the two different tenor curves} To back out a Libor (e.g. 3m) and funding (e.g. Fed-Funds) curve, we could consider a set of $N$ collateralized swaps (so funding is at Fed-Funds), and a set of $M$ collateralized  Fed-Funds vs 3m Libor tenor swaps (again funding at Fed-Funds) - the swaps are mostly sensitive to the Libor curve, so the $N$ nodes on this curve should be at the swap maturities, while the FRNs are mainly sensitive to the spread between the Libor and Fed-fund curve, in which case the $M$ nodes of the fund curve should be at the maturity of the FRNs. This treats both curves on an equal footing. An alternative is to have a primary Libor curve and a spread curve (funding is Libor - spread) - the Libor curve could use more than $N$ nodes, with the remaining going to the spread curve.\footnote{The point is that the two curves will most likely have very similar macro features that you want to capture with as many nodes as possible, while the spread could be a very simple thing represented by only a few nodes (or one for a constant spread).  The funding curve could equally take the role of the primary curve.}

\paragraph{Cross Currency Swaps (CCS)}
A CCS is effectively an exchange of a domestic FRN on a notional of $N$ for a foreign FRN on a notional of $Nx(0)$ where $x(0)$ is the spot forex rate (units of foreign currency for 1 domestic). If both legs have spreads set up as in section \ref{sec:FRN}, then they will both PV to zero and can be treated as separate FRNs (one being sensitive to domestic funding and a domestic index curve, and the other being sensitive to foreign funding and a foreign index curve).

In reality, a single spread is paid on one of the legs, which renders the whole contract fair. In general neither leg is separately PV zero. Of course, if the funding and index curve are the same for one leg, we can treat the other leg as a FRA provided it is the one that the spread is paid on.

Let $p^d(t)$ and $p^f(t)$ be domestic and foreign discount factors for the respective funding curves and $F^d_i$ and $F^f_i$ be the domestic and foreign index rates which are paid at times $t^d_1\ldots t^d_{N^d}$ and  $t^f_1\ldots t^f_{N^f}$. The PV on the domestic side (i.e. paying the domestic index rate plus spread and receiving the foreign) for unit notional is:
%
\begin{equation}
\begin{split}
PV =  \frac{x(t_0)}{x(0)}\left[-p^f(t_0) + p^f(t^f_{N^f}) +\sum^{N^f}_{i=1} F^f_i \alpha^{f}_ip^{f}(t^{f}_i)\right]\\
-\left[-p^d(t_0) + p^d(t^d_{N^d}) +\sum^{N^d}_{i=1}( F^d_i+s) \alpha^{d}_ip^{d}(t^{d}_i)\right]
\end{split}
\end{equation}
%
The term $\frac{x(t_0)}{x(0)}$ arrises from exchanging notional amounts on the trade date $t_0$ (normally two days) and converting the foreign cash flow back to domestic now. The spread that makes this fair is
%
\begin{equation}
s =  \frac{\frac{x(t_0)}{x(0)}\left[-p^f(t_0) + p^f(t^f_{N^f}) +\sum^{N^f}_{i=1} F^f_i \alpha^{f}_ip^{f}(t^{f}_i)\right]+p^d(t_0) -p^d(t^d_{N^d}) -\sum^{N^d}_{i=1}F^d_i \alpha^{d}_ip^{d}(t^{d}_i)}
{\sum^{N^d}_{i=1}\alpha^{d}_ip^{d}(t^{d}_i)}
\end{equation}
%
The PV (and equivalently the spread) is sensitive to (up to) four curves\footnote{the funding in both currencies, and the Libors in both currencies} - these sensitivities (and that for the spread) can be calculated in the same manner as the previous sections. 

\paragraph*{Disscusion}
The tenors used in CCS do not necessarily match those of interest rate swaps (IRS), so one could easily be in a situation of needing strips of foreign and domestic IRSs (plus other instruments to take care of the short end of the curve), strips of overnight rates versus Libor basis swaps, Libor tenor swaps, and CCSs. In this situation one would most likely want base curves of the foreign and domestic funding rates, then all other curves implemented as spreads over these curves.

\section{Hedging}
Let $x_1\ldots x_N$ be the PV of the instruments\footnote{we will use $x_i$ to refer to both an instrument and its price} used to build the curve(s), and  $r_1\ldots r_N$ be the yields at the node points on the curve(s)\footnote{the distinction between one and many curves is not relevant here, so we will just talk above the yields $r_i$ which could refer to one or many curves}. The Jacobian is
\begin{equation}
\mathbf{J} = \left(  
\begin{array}{ccc}
\frac{\partial x_{1}}{\partial r_1} & \ldots & \frac{\partial x_1}{\partial r_N}\\
\vdots&\ddots&\vdots\\
\frac{\partial x_N}{\partial r_1}&\ldots&\frac{\partial x_N}{\partial r_N}
\end{array}
\right)
\end{equation}
%
The sensitivity of the instruments to a parallel movement in the curve is given by summing each row. Mathematically this is 
\[
\frac{\partial \mathbf{x}}{\partial r}|_{\text{parallel move}} = \mathbf{J}\mathbf{1} \quad \text{where} \quad \mathbf{1} = \left(  
\begin{array}{c}
1\\
\vdots\\
1
\end{array}
\right)
\]
Sensitivities to other deformations of the curve can be found by replacing the vector $\mathbf{1}$ with one describing the appropriate deformation (e.g. steepening of the curve).
\subsection{Delta Hedging a Curve Sensitive Instrument}
Let $y$ be some other instrument whose price depends purely on the curve we have implied from the $N$ instruments $x_i$ above. A portfolio can be formed thus:
%
\begin{equation}
\Pi = y-\sum^N_{i=1}w_ix_i
\end{equation}
%
where the weights $w_i$ are the amounts held of $x_i$ (hold one unit of $y$). To be insensitive to any small movement of the curve requires 
%
\begin{equation}
\frac{\partial \Pi}{\partial r_k}=0 \quad \Rightarrow \quad  \frac{\partial y}{\partial r_k} =\sum^N_{i=0}w_i\frac{\partial x_i}{\partial r_k} \quad \forall k
\end{equation}
%
Writing the weights $w_i$ and the curve sensitivity of $y$\footnote{The quantities $ \frac{\partial y}{\partial r_k}$ are calculated by the system exactly as the entries in the Jacobian are} in  vector form 
\begin{equation}
\mathbf{w} = \left(  
\begin{array}{c}
w_1\\
\vdots\\
w_N
\end{array}
\right)
\quad
\mathbf{y}' = \left(  
\begin{array}{c}
\frac{\partial y}{\partial r_1 }\\
\vdots\\
\frac{\partial y}{\partial r_N}
\end{array}
\right)
\end{equation}
%
gives 
\begin{equation}
\mathbf{y}'=\mathbf{J}^T\mathbf{w} \quad \Rightarrow \quad \mathbf{w} =\left( \mathbf{J}^{-1}\right)^T\mathbf{y}'
\end{equation}
%
In general every entry in $\mathbf{w}$ will have a non-zero value, implying some position will need to be taken in every instrument. Furthermore there is no guarantee that the weights will not contain large positive and negative values\footnote{this is often seen in mean-variance portfolio optimisation, where mathematically the equation for the weights is the same}, making the hedge extremely sensitive to non-negligible movements of the curve (i.e large gamma).
In practice one wants to pick a few hedging instrument that are sensitive to the curve and match the properties of $y$ (e.g. similar maturity, etc). Let the M instruments be $h_1\ldots h_M$, and again form a portfolio 
%
\begin{equation}
\Pi = y-\sum^M_{i=1}w_ih_i
\end{equation}
%
With $M<N$ it is not possible to meet the condition $\frac{\partial \Pi}{\partial r_k}=0$ for all $k$. We define the sensitivity matrix $\mathbf{H}$ and the vector of small movements of the yields, $\mathbf{\Delta r}$ as
\begin{equation}
\mathbf{H} = \left(  
\begin{array}{ccc}
\frac{\partial h_{1}}{\partial r_1} & \ldots & \frac{\partial h_1}{\partial r_N}\\
\vdots&\ddots&\vdots\\
\frac{\partial h_M}{\partial r_1}&\ldots&\frac{\partial h_M}{\partial r_N}
\end{array}
\right)
\quad
\mathbf{\Delta r} = \left(  
\begin{array}{c}
\Delta r_1\\
\vdots\\
\Delta r_N
\end{array}
\right)
\end{equation}
%
so a small change in the value of the portfolio can be written as
\begin{equation}
\Delta \Pi \approx - \mathbf{\Delta r} \cdot (\mathbf{H}^T\mathbf{w}-\mathbf{y}')
\end{equation}
This is a random variable, which depends on the random vector $\mathbf{\Delta r}$. To keep things tractable we aim to minimise the expectation of the square of this value:
\begin{equation}
\begin{split}
\mathbb{E}[\Delta \Pi^2] = (\mathbf{H}^T\mathbf{w}-\mathbf{y}')^T \mathbf{\Theta} (\mathbf{H}^T\mathbf{w}-\mathbf{y}')\\
\text{where}  \quad\mathbf{\Theta}  = \mathbb{E}[ \mathbf{\Delta r} \mathbf{\Delta r}^T]
\end{split}
\end{equation}
So $\mathbf{\Theta}$ should be available from time series data of the changes in the yields - if $\mathbb{E}[\Delta \Pi]=0$ (i.e. there is no drift), then $\mathbf{\Theta}$ is the covariance matrix of changes in yields. Differentiating with respect to the weights we have
 %
\begin{equation}
\begin{split}
\frac{\partial \mathbb{E}[\Delta \Pi^2]}{\partial \mathbf{w}} = \mathbf{0} \Rightarrow\\
 \mathbf{H}^T\mathbf{\Theta} \mathbf{H} \mathbf{w}= \mathbf{H}^T\mathbf{\Theta}\mathbf{y}' \Rightarrow\\
 \mathbf{w}=( \mathbf{H}^T\mathbf{\Theta} \mathbf{H} )^{-1} \mathbf{H}^T\mathbf{\Theta}\mathbf{y}' 
\end{split}
\end{equation}
This will in general give a more robust hedge to movements of the curve.

\section{Conclusion}
Any interest rate product with price sensitivity to a set (possibly only one) of yield curves only, can be used to imply these curves provided one can write down a function from yield curves to price.  If when attempting to imply several curves, there are no instruments that are sensitive to all the curves, then the problem decouples and the different curves can be found separately.


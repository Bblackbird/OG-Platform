\section{Introduction}

This document covers how to write documentation for OpenGamma code using Dexy.

See the building-docs.tex document for how to run Dexy and build this
documentation.

As a prerequisite to reading this, you should read the page on
\href{http://www.dexy.it/docs/guide/documents-dependencies-and-filters/}{Documents, Dependencies and Filters}
from the \url{http://dexy.it} website which describes running Dexy, writing
\verb|.dexy| files and using filters.

\section{Dexy Configuration}

The examples directory has a \verb|.dexy| file which should cover many basic
cases. In Dexy, the configuration in a .dexy file is applied to the directory
it's in, along with any subdirectories (unless instructed otherwise). You can
put an additional \verb|.dexy| file in your example directory, but you
shouldn't need to for straightforward examples.

Here is the .dexy file in the examples directory:

<< d['../.dexy|ppjson|pyg|l'] >>

So, among other things, we can see that all \verb|.java| files will be run
through the fn filter, then the java filter. The fn filter helps manage
filenames for data files or images. It is described in
\href{http://www.dexy.it/docs/filters/FilenameFilter-(fn).html}{this filter reference page}.
The java filter will compile and then run the code. It is described in
\href{http://www.dexy.it/docs/filters/JavaFilter-(java).html}{this filter reference page}.

The OpenGammaExample.java class is run through the .javac filter (which just
compiles the code without running it), and this is an input to all other java
files. This means that OpenGammaExample.class is placed on the classpath when
all other java files are run. We will discuss how to use this class for writing
examples later on.

\section{Hello, World Examples}

Examples form the basis for documenting features and also for creating plots
and other visual assets to be used in white papers. In this section, we create
some very simple examples to introduce basic Dexy concepts.

Example .java files should be placed in subdirectories of the \verb|docs/examples|
directory. You can write an index.md or index.html file to document your example.

\subsection{Hello, World Example}

We'll write a short self-contained script that uses a simple OpenGamma class.

Here is the script:

%%% @export "hello-og-world-source"
<< d['hello-og-world/HelloOpenGammaWorldExample.java|pyg|l'] >>
%%% @end

And here is the output from running the script via dexy:

%%% @export "hello-og-world-output"
\begin{Verbatim}
<< d['hello-og-world/HelloOpenGammaWorldExample.java|fn|java'] >>
\end{Verbatim}
%%% @end

The point of this script is that (a) it's very simple and (b) it connects to an
OpenGamma class so we know our CLASSPATH etc. is working.

Here is the file in the \verb|hello-og-world| directory:

\begin{Verbatim}
<< d['hello-og-world/list-files.sh|jinja|sh'] >>
\end{Verbatim}

The syntax highlighted source code can be pulled into a LaTeX document by:

<< d['writing-docs.tex|idio|l']['hello-og-world-source'] >>

The \verb=|l= filter at the end forces the previous filter to output \LaTeX. If
you are writing a HTML document you would leave this off (and use \lt pre \gt tags to
display plain text instead of creating a Verbatim environment).

The output from running this example can be pulled in by:

<< d['writing-docs.tex|idio|l']['hello-og-world-output'] >>

The \verb|java| filter compiles and runs java code, and returns whatever gets written to STDOUT.

\subsection{Longer Hello, World Example}

Now we'll have a slightly longer example in order to demonstrate how we split
our code into sections.

Here is the complete source code of the example:

%%% @export "hello-og-world-longer-source"
<< d['hello-og-world-longer/HelloOpenGammaWorldLongerExample.java|pyg|l'] >>
%%% @end

The comments starting with \verb|// @export| are special comments recognized
by the \verb|idio| filter which split the source code up into sections. We can
pull these sections in to documents individually, so we can discuss sections of
the code in detail.

For example, here is the `imports' section where we list the packages we wish to import:

%%% @export "hello-og-world-longer-source"
<< d['hello-og-world-longer/HelloOpenGammaWorldLongerExample.java|idio|l']['imports'] >>
%%% @end

Here is how that section is pulled in to this document:

<< d['writing-docs.tex|idio|l']['hello-og-world-longer-source'] >>

Next we have our class definition, and we start the \verb|main()| method:

<< d['hello-og-world-longer/HelloOpenGammaWorldLongerExample.java|idio|l']['class-def'] >>

We initalize a complex number:

<< d['hello-og-world-longer/HelloOpenGammaWorldLongerExample.java|idio|l']['init'] >>

And we do some trigonometric calculations:

<< d['hello-og-world-longer/HelloOpenGammaWorldLongerExample.java|idio|l']['calculate'] >>

Here is the output:

%%% @export "hello-og-world-longer-output"
\begin{Verbatim}
<< d['hello-og-world-longer/HelloOpenGammaWorldLongerExample.java|fn|java'] >>
\end{Verbatim}
%%% @end

Dividing this code up into sections allows us to explain each part in detail,
without overwhelming the reader with an entire script.

\subsection{Saving Data}

In this section we will write a script that writes data to a file. We will then
be able to show that file in our document.

Here are the imports and class declaration:

<< d['hello-og-world-data/HelloOpenGammaWorldLongerExample.java|idio|l']['imports'] >>
<< d['hello-og-world-data/HelloOpenGammaWorldLongerExample.java|idio|l']['class-def'] >>

We open a data file and initialize a PrintStream to write to this file:

<< d['hello-og-world-data/HelloOpenGammaWorldLongerExample.java|idio|l']['open-data-file'] >>

By starting the file name with \verb|dexy--| we are telling the Filename (fn)
filter to replace that canonical filename with a special filename which will
help us share our data with other documents. You don't need to worry about the
details, just keep in mind that if you want to write data to a file so it can
be used later or included in your documents, you should start the filename with
\verb|dexy--| and run your script through the \verb|fn| filter before you run
the code.

Java is a special case in Dexy. Normally, Dexy runs scripts by saving them in
the \verb|artifacts/| directory under a special filename. However, Java
requires that classes are defined in files with a matching name. So, for Java,
Dexy creates a directory with a special name and saves the script in that
directory under its canonical name. However, when we use the filename filter to
create a data file, it expects the data file to be created in the artifacts
directory rather than a subdirectory. So, for Java, you need to prefix your
data file name with \verb|..| so that the data file is written to the correct
location.

Now that we have our print stream set up, we can write data to the file:

<< d['hello-og-world-data/HelloOpenGammaWorldLongerExample.java|idio|l']['init'] >>
<< d['hello-og-world-data/HelloOpenGammaWorldLongerExample.java|idio|l']['calculate'] >>

Finally, we close the data stream:

<< d['hello-og-world-data/HelloOpenGammaWorldLongerExample.java|idio|l']['close-data-file'] >>

Here is what was written to the data file:

%%% @export "hello-og-world-data-file"
\begin{Verbatim}
<< d['hello-og-world-data/hello-world-data.txt'] >>
\end{Verbatim}
%%% @end

And here is how we pulled that into this document:

<< d['writing-docs.tex|idio|l']['hello-og-world-data-file'] >>

You can use the filename (fn) filter to help manage any type of file. Typically
we will create .txt or .csv files (it could even be a .sqlite3 file) to share
data between a script which generates the data, and another which analyzes or
plots the data (which can be in another programming language). We will also use
the filename filter to manage the .png or .pdf filenames that the plots go in.

\section{The OpenGammaExample Class}

The OpenGammaExample class is intended to make it easier to create Java classes
with several related example methods, run each of these methods automatically,
and be able to reference the output from each method.

In the next section, we show what you can do with this class. In the following
section, we look at how the class is implemented.

\subsection{SimpleExample}

Here is the source in full of the example we will be discussing:

<< d['using-opengamma-example-class/SimpleExample.java|pyg|l'] >>

Any method in this class that accepts a single argument of class PrintStream
will be automatically run, and the contents written to the PrintStream will be
available under that method's name.

Idiopiade-style comments are used to divide the code into sections, so
you can show a section:

<< d['using-opengamma-example-class/SimpleExample.java|idio|l']['basicDemo'] >>

And then show the output generated by that section:

%%% @export "show-output"
\begin{Verbatim}
<< d['using-opengamma-example-class/simple-output.json']['basicDemo'] >>
\end{Verbatim}
%%% @end

Here is the code used to pull that output into this document:

<< d['writing-docs.tex|idio|l']['show-output'] >>

%%% @export "show-fields"
Any class fields are also available, so you can refer to the string HELLO, i.e.
'<< d['using-opengamma-example-class/simple-fields.json']['HELLO'] >>', in your document.
%%% @end

Here is the code used to pull the field value into that sentence:

<< d['writing-docs.tex|idio|l']['show-fields'] >>

If you need to share code among multiple demos, for example if you want to instantiate an object with the same parameters in several examples, then you can create a helper method for it, show it once (if necessary), and call it several times.

Here is a helper method:

<< d['using-opengamma-example-class/SimpleExample.java|idio|l']['helperMethod'] >>

We use the helper method in this example:

<< d['using-opengamma-example-class/SimpleExample.java|idio|l']['useHelperMethod'] >>

\begin{Verbatim}
<< d['using-opengamma-example-class/simple-output.json']['useHelperMethod'] >>
\end{Verbatim}

The main method in our example looks like this:

<< d['using-opengamma-example-class/SimpleExample.java|idio|l']['main'] >>

We simply call the OpenGammaExample class's main method and pass 3 parameters.
The first is the name of the example class, the second is a file name for
storing the output written by the examples, the third (optional) is a filename
for storing field names and values.

Note we use the \verb|dexy--| prefix so that Dexy handles dependencies for
these generated files.

\subsection{The OpenGammaExample Class}

In this section we look at the source of the OpenGammaExample class and how it
works.

We saw that in our example, we call the OpenGammaExample class's main method.
Here is the start of this method:

<< d['../OpenGammaExample.java|idio|l']['main'] >>

We process the args array, assigning defaults if no args are passed (i.e. if
the OpenGammaExample class itself is being run).

Next we set up some output streams we will use to save output from the methods
we call:

<< d['../OpenGammaExample.java|idio|l']['outputStreams'] >>

Next, we do most of the 'work' of this method. Identifying the methods we want
to run and running them:

<< d['../OpenGammaExample.java|idio|l']['findAndRunMethods'] >>

This uses Java reflection to obtain a list of methods, and it iterates over
them, calling \verb|runMethod| on any that meet the criteria:

<< d['../OpenGammaExample.java|idio|l']['runMethod'] >>

The \verb|outputStreams| object is a JSONObject which stores key-value pairs.
After all methods have been run, the output collected in outputStreams is
written to the JSON file which was passed as the 2nd entry to ogargs.

<< d['../OpenGammaExample.java|idio|l']['saveOutput'] >>

Finally, if a 3rd argument has been passed, then we also use reflection to pull
the example class's declared fields and their values and save them in another
JSON file.

<< d['../OpenGammaExample.java|idio|l']['saveFields'] >>

\section{Referencing Source Code}

Up to now, we have looked at creating small, self-contained examples
which make use of OpenGamma classes. However, we might also want to pull in
extra information about those classes, such as Javadoc comments or the actual
source code of the methods.

In this section, we will look at how to pull that information in to a document.

In the next section, we will go through writing an extensive example in Java
and also writing up documentation for the example, including pulling in the
source code, the results, and documentation for the classes we reference.

The json-doclet project is part of Dexy, and it lets us run a custom Javadoc
\href{http://download.oracle.com/javase/6/docs/technotes/guides/javadoc/doclet/overview.html}{Doclet}
which gathers all the Javadoc information plus extracts source code, and
writes all this information to a JSON file.

In theory, any element from the \href{http://download.oracle.com/javase/6/docs/jdk/api/javadoc/doclet/index.html}{Doclet API}
can be made available, although not all elements are implemented yet.

Within our Dexy documents, we can pull information out of this JSON dict. A
special Dexy filter adds some nice features like syntax highlighting the Java
source code stored in the dict. You can view the filter reference documentation
for the \href{http://dexy.it/docs/filters/javadoc}{javadoc filter}.

%%% @export "assign-class-to-local-variable"
<% set cn_class = d['/jsondocs/javadoc-data.json|javadocs']['packages']['com.opengamma.math.number']['classes']['ComplexNumber'] %>
%%% @end

When working with a class, it can be useful to assign that branch of the dict to a local variable:

\fvset{fontsize=\scriptsize}
<< d['writing-docs.tex|idio|l']['assign-class-to-local-variable'] >>
\fvset{fontsize=\small}

Then we can access class comment text via:

<< d['writing-docs.tex|idio|l']['class-comment-text'] >>

Here is the comment for this class:

%%% @export "class-comment-text"
<< cn_class['comment-text'] >>
%%% @end

We can iterate over methods in the class docs via:

<< d['writing-docs.tex|idio|l']['print-methods'] >>

Documentation is available for the following methods:

%%% @export "print-methods"
\begin{itemize}
<% for name in cn_class['methods'].keys() -%>
\item{<< name >>}
<% endfor -%>
\end{itemize}
%%% @end

We can iterate over constructors in the class docs via:

<< d['writing-docs.tex|idio|l']['print-constructors'] >>

Documentation is available for the following constructors:

%%% @export "print-constructors"
\begin{itemize}
<% for name in cn_class['constructors'].keys() -%>
\item{<< name >>}
<% endfor -%>
\end{itemize}
%%% @end

In the previous section, we used the following constructor:

<< cn_class['constructors']['ComplexNumber(double,double)']['source-latex'] >>

If we didn't have an imaginary component, we could have used an alternate constructor:

\begin{Verbatim}
<< cn_class['constructors']['ComplexNumber(double)']['comment-text'] >>
\end{Verbatim}
<< cn_class['constructors']['ComplexNumber(double)']['source-latex'] >>

You can use Python to extract just the first line of the source code:

<< d['writing-docs.tex|idio|l']['extract-first-line-source'] >>

%%% @export "extract-first-line-source"
<< cn_class['constructors']['ComplexNumber(double)']['source-latex'].splitlines()[0] >>
<< cn_class['constructors']['ComplexNumber(double)']['source-latex'].splitlines()[1] >>
<< cn_class['constructors']['ComplexNumber(double)']['source-latex'].splitlines()[-1] >>
%%% @end

Line 0 will have the opening Verbatim tag, and the final line will have the closing Verbatim tag. Line 1 will have the first line of highlighted code.

Or, use the plain text rather than highlighted version, and add your own Verbatim tags:

<< d['writing-docs.tex|idio|l']['extract-first-line-source-alt1'] >>

%%% @export "extract-first-line-source-alt1"
\begin{Verbatim}
<< cn_class['constructors']['ComplexNumber(double)']['source'].splitlines()[0] >>
\end{Verbatim}
%%% @end

You can also strip off the curly bracket:

<< d['writing-docs.tex|idio|l']['extract-first-line-source-alt2'] >>

%%% @export "extract-first-line-source-alt2"
\begin{Verbatim}
<< cn_class['constructors']['ComplexNumber(double)']['source'].splitlines()[0].rstrip("{ ") >>
\end{Verbatim}
%%% @end

Or, you can retrieve the \href{http://download.oracle.com/javase/6/docs/jdk/api/javadoc/doclet/com/sun/javadoc/ExecutableMemberDoc.html#signature()}{signature}:

\begin{Verbatim}
<< cn_class['constructors']['ComplexNumber(double)']['signature'] >>
\end{Verbatim}

Here is the source of the \verb|toString()| method:

<< cn_class['methods']['toString()']['source-latex'] >>

\subsection{Class Information}

Here are the methods available for this class:

\begin{itemize}
<% for key, value in cn_class.iteritems() -%>
\item{<< key >>}
<% endfor -%>
\end{itemize}

\subsection{Constructor Information}

<% set cn_constructor = "ComplexNumber(double)" -%>

Here are the methods available for the << cn_constructor >> constructor:

\begin{itemize}
<% for key, value in cn_class['constructors'][cn_constructor].iteritems() -%>
\item{<< key >>}
<% endfor -%>
\end{itemize}

Here are some examples of what these items return:

<% for key in ["signature", "flat-signature", "modifiers", "qualified-name"] -%>
\subsubsection{<< key >>}

\begin{Verbatim}
<< cn_class['constructors'][cn_constructor][key] >>
\end{Verbatim}

<% endfor -%>

\subsection{Method Information}

<% set cn_method = "toString()" -%>

Here are the methods available for the << cn_method >> method:

\begin{itemize}
<% for key, value in cn_class['methods'][cn_method].iteritems() -%>
\item{<< key >>}
<% endfor -%>
\end{itemize}

\section{A Real Example}

In this section, we will work through a real example which makes use of all the
techniques we have been developing.

The SABR extrapolation example is adapted from a unit test. This example shows
how to make use of some OpenGamma classes, and it also generates a data file.
We will then write an R script to read this data and generate plots.

<% set sabr_class = d['/jsondocs/javadoc-data.json|javadocs']['packages']['com.opengamma.financial.model.option.pricing.analytic.formula']['classes']['SABRExtrapolationRightFunction'] %>

First we look at some of the docs and source code of the classes we will use in
our example.

\subsection{SABRExtrapolationRightFunction Class}

<< sabr_class['comment-text'] >>

We will want to create an instance of this function. Here is the constructor:

\fvset{fontsize=\footnotesize}
<< sabr_class['constructors']['SABRExtrapolationRightFunction(double,SABRFormulaData,double,double,double)']['source-latex'] >>
\fvset{fontsize=\small}

Then we will use the \verb|price| function.

<< sabr_class['methods']['price(EuropeanVanillaOption)']['comment-text'] >>
\fvset{fontsize=\footnotesize}
<< sabr_class['methods']['price(EuropeanVanillaOption)']['source-latex'] >>
\fvset{fontsize=\small}

\subsection{SABRFormulaData}

The other class we use is SABRFormulaData.

<% set sabr_formula_data_class = d['/jsondocs/javadoc-data.json|javadocs']['packages']['com.opengamma.financial.model.volatility.smile.function']['classes']['SABRFormulaData'] %>

<< sabr_formula_data_class['constructors']['SABRFormulaData(double,double,double,double)']['source-latex'] >>

\subsection{Example}

Here are the OpenGamma classes that we import at the start of our example:

<< d['../sabr-extrapolation/SabrExtrapolationExample.java|idio|l']['imports'] >>

We also import some other utility classes:

<< d['../sabr-extrapolation/SabrExtrapolationExample.java|idio|l']['util-imports'] >>

Here is our class definition:

<< d['../sabr-extrapolation/SabrExtrapolationExample.java|idio|l']['class-def'] >>

We define constants used in the SABRFormulaData constructor, and then create a
\verb|SABR_DATA| constant:

<< d['../sabr-extrapolation/SabrExtrapolationExample.java|idio|l']['class-constants-sabr-data'] >>

Next we set some more constants we will use in our calculations:
<< d['../sabr-extrapolation/SabrExtrapolationExample.java|idio|l']['class-constants-sabr-extrapolation-function'] >>

We will make use of the OpenGammaExample class, so we create a method which
accepts a single PrintStream argument. We could also just use the \verb|main|
method and not bother with using OpenGammaExample since we are only writing a
single example method, however by using OpenGammaExample we make it easier to
refactor this later, if we decide we do want to add another related example.

Also, by using OpenGammaExample, all our class fields will automatically be
saved so we can easily say in documents that
$\alpha=<< d['../sabr-extrapolation/sabr-extrapolation-fields.json']['ALPHA'] >>$ and
$\beta=<< d['../sabr-extrapolation/sabr-extrapolation-fields.json']['BETA'] >>$.

Here is the start of our method, we declare some local variables and set up
some formulas and data bundles we will use later on:

<< d['../sabr-extrapolation/SabrExtrapolationExample.java|idio|l']['generateSabrData'] >>

We set up the tab-separated data file we will store our data in, and write a
header row so we know the field names:

<< d['../sabr-extrapolation/SabrExtrapolationExample.java|idio|l']['data-file'] >>

Now we can enter our main loop. We loop over different values of mu, setting up
a SABRExtrapolationRightFunction for each value, then we loop over various
strike prices and calculate option prices and implied volatility:

\fvset{fontsize=\scriptsize}
<< d['../sabr-extrapolation/SabrExtrapolationExample.java|idio|l']['loop'] >>
\fvset{fontsize=\small}

Finally, we close the data stream, ensuring all our data is written to a file:

<< d['../sabr-extrapolation/SabrExtrapolationExample.java|idio|l']['save-data'] >>

Here is the main method that calls the OpenGammaExample class's main method:

<< d['../sabr-extrapolation/SabrExtrapolationExample.java|idio|l']['main'] >>

Here are the first few lines of the tab-separated data file we generated:

\begin{Verbatim}
<< "\n".join(d['../sabr-extrapolation/smile-multi-mu-data.txt'].splitlines()[0:10]) >>
\end{Verbatim}

\subsection{Plotting Data in R}

The reason we have generated this data is to make use of it. In this case, we
want to plot this data in R. The examples repository is configured so that if
you create a .R file with the same name as the .java file, this R file will
be run automatically, after the .java file has been compiled and run, so it can
make use of any data files generated by the java.

At the top of our R script, we state which libraries we will need:

<< d['../sabr-extrapolation/SabrExtrapolationExample.R|idio|l']['libraries'] >>

Next we read the generated data into a variable creatively named \verb|data|:

<< d['../sabr-extrapolation/SabrExtrapolationExample.R|idio|l']['read-data'] >>

We need to calculate the density separately for each value of Mu. To simplify
this, we begin by creating variables to hold Price and Strike for different
values of Mu.

<< d['../sabr-extrapolation/SabrExtrapolationExample.R|idio|l']['extract-data-by-mu'] >>

Next we calculate density for each of the values of Mu, and add the
concatenated vectors back into our original data object.

<< d['../sabr-extrapolation/SabrExtrapolationExample.R|idio|l']['calculate-density'] >>

Now we are ready to plot. To start with, we plot the option price against the strike price:

<< d['../sabr-extrapolation/SabrExtrapolationExample.R|idio|l']['plot-data-pdf'] >>

%%% @export "include-extrapolation-price-pdf"
\includegraphics{<< a['../sabr-extrapolation/extrapolation-price.pdf'].filename() >>}
%%% @end

Because we are working in \LaTeX, we use the \verb|includegraphics| commands
and reference the dexy artifact corresponding to the plot we want to include,
and call that artifact's \verb|filename()| method:

<< d['writing-docs.tex|idio|l']['include-extrapolation-price-pdf'] >>

By the time we run \LaTeX, our expression has been replaced with the name of
the PDF file we want to include.

Here is the second plot, of the implied volatility against the strike price:

<< d['../sabr-extrapolation/SabrExtrapolationExample.R|idio|l']['plot-implied-vol-pdf'] >>

\includegraphics{<< a['../sabr-extrapolation/extrapolation-smile.pdf'].filename() >>}

And the third plot, of the density against the strike price:

<< d['../sabr-extrapolation/SabrExtrapolationExample.R|idio|l']['plot-density-pdf'] >>

\includegraphics{<< a['../sabr-extrapolation/extrapolation-density.pdf'].filename() >>}

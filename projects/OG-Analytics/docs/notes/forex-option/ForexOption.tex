\documentclass[]{amsart}

\usepackage{amsmath}
\usepackage{amssymb}
\usepackage{color}
\usepackage[
   pagebackref=true,
   colorlinks=true,
   urlcolor=ogblue,
   citecolor=oggreen,
   linkcolor=ogblue]{hyperref}
%\usepackage{epsfig}
%\usepackage{subfigure}
\usepackage{natbib}
\usepackage{pgf}

\pgfdeclareimage[height=0.7cm]{oglogo}{../../shared/OpenGammaLogo}

\definecolor{ogblue}{RGB}{0,85,129}
\definecolor{oggreen}{RGB}{93,194,164}
\definecolor{ogbluelight}{RGB}{17,116,158} 
\definecolor{ogred}{RGB}{204,44,0}
\definecolor{oggray}{RGB}{178,178,178} 
\definecolor{oggraylight}{RGB}{234,234,234}

\setlength{\oddsidemargin}{0.5cm}
\setlength{\evensidemargin}{0.5cm}
\setlength{\textwidth}{15cm}
\setlength{\textheight}{22cm}

\newcommand{\TODO}[1]%
   {{\sffamily \textbf{\color{oggreen}\noindent TO DO}\marginpar{\hspace*{0.5cm}
     \sffamily \textbf{\color{oggreen}TO DO}}: #1}}
     
\newtheorem{theo}{Theorem}
     
\newcommand{\rem}{\par\noindent \textit{Remark:} }
\newcommand{\class}[1]{{\texttt{#1}}}
\newcommand{\spot}{{\operatorname{Spot}}}
\newcommand{\pv}{{\operatorname{PV}}}
\newcommand{\rr}{{\operatorname{RR}}}
\newcommand{\str}{{\operatorname{SS}}}
\newcommand{\atm}{{\operatorname{ATM}}}
\newcommand{\E}[2]{\operatorname{E}^{#1}\left[#2\right]}
%\newcommand{\N}{{\mathbb N}}
     
\title[Forex options]%
   {Forex option}

\author[M. Henrard]%
   {Marc Henrard}

\date{First version: 10 June 2011; this version: 21 June 2011}

\thanks{Version 1.1}

\address{Marc Henrard \\ Quantitative Research \\ OpenGamma}

\email{marc@opengamma.com}

\begin{document}

\maketitle

\begin{center}
\pgfuseimage{oglogo} 
\end{center}

\begin{abstract}
Pricing of vanilla Forex options using a Black formula and a delta and time dependent volatility.
\end{abstract}

\section{Introduction}

The note presents the pricing of vanilla Forex European options. The data is described by the standard figures in the Forex market with the smile represented by the ATM volatility, risk reversals and strangles.

A Forex transaction is represented by an amount in foreign currency $N_1$ and amount in domestic currency $N_2$ (also called quote currency) and a payment date $t_p$ ($N_1$ and $N_2$ have opposite signs). The option on the forex transaction is represented by the underlying Forex transaction, a strike $K$, the expiry $\theta \leq t_p$ and a call/put feature. The strike is relate to the amounts by $K = -N_2/N_1$. 

There are two currencies involved and thus two discounting curves. They are denoted $P_D$ for the domestic currency and $P_F$ for the foreign currency. The $P_F$ used here should not be confused with the $P^F$ used in interest rate modelling where the $F$ refers to forward curves.

For a EURUSD transaction, buying EUR at a rate of 1.47 and for a nominal of 1,000,000 EUR, the domestic currency is USD and the foreign currency is EUR. The amounts are $N_1=1,000,000$ and $N_2=-1,470,000$.

\section{Preliminaries}

In this note (and in the implementation), we consider that the exchange rate used is today's exchange rate (and not spot exchange rate). 

The options present value are computed from a Black formula with strike and expiry dependent volatility. The forward rate for (payment) date $t$ and today rate $S_0$ is given by
\[
F^{t}_0 = \frac{P_F(0,t)}{P_D(0,t)} S_0.
\]

The present value is computed in the domestic currency.
The present value with the Black formula and a volatility $\sigma$ is
\[
\pv = N_1 P_D(0,t_p) \omega \left( F^{t_p}_0 N(\omega d_+) - K N(\omega d_-)\right)
\]
where $\omega = 1$ for a call, $\omega = -1$ for a put and 
\[
d_{\pm} = \frac{\ln\left(\frac{F^{t_p}_0}{K}\right) \pm \frac12 \sigma^2 t}{ \sigma \sqrt{t}}.
\]

The delta of the forward value with respect to the forward rate is given by
\begin{equation}
\label{EqnDeltaForward}
\Delta_F = \omega N(\omega d_+).
\end{equation}
The delta of the present value with respect to the today's rate is given by
\[
\Delta_\spot = \omega P_F(0,t) N(\omega d_+).
\]

\section{Smile}

The smile is the description of the implied volatility for each strike, i.e. of a volatility function $\sigma = \sigma(K)$.
A standard smile description is done with the at-the-money (ATM) volatility, risk reversals (RR) and smile strangles (SS) (see \cite{CLA.2011.1} for more details and other possibilities).

The delta used here is the delta forward described in (\ref{EqnDeltaForward}).
The \emph{at-the-money} (ATM) used is the ATM - Delta Neutral Straddle (DNS), i.e. the strike for which the straddle has 0 delta forward. It is denoted $K_\atm$.

The risk reversals and strangles are given for some deltas, in general for delta 0.10 and 0.25. It means that the figures given are related to puts with deltas -0.10 and -0.25 and to calls with deltas 0.25 and 0.10. All the options are out-of-the-money; the puts have strikes below the ATM strikes and calls have strikes above the ATM strike. The strikes are denoted $K_{x,T}$ with $x$ the absolute value of the delta and $T$ the type ($C$ for call and $P$ for put).

The risk reversals contain information about the skew (or slope) of the smile. The risk reversal figure is the difference between the volatility at the call and the volatility at the put:
\[
\rr(x) = \sigma(K_{x,C}) - \sigma(K_{x,P}).
\]

The strangles contain information about the curvature of the smile. The strangle figure is the difference between the average of the volatility out-of-the-money and volatility at-the-money:
\[
\str(x) = \frac12 \left(\sigma(K_{x,C}) + \sigma(K_{x,P})\right) - \sigma(K_{\atm}).
\]


The strike can be computed explicitly from the forward delta by
\begin{equation}
\label{EqnDelta}
K = F \exp\left( -(\sigma \sqrt{t} \omega N^{-1}(\omega \Delta) - \frac12 \sigma^2 t)  \right).
\end{equation}

The smile is described with the figures
\begin{enumerate}
\item Deltas (usually 0.25 and 0.10).
\item ATM volatility.
\item Risk reversal for each delta.
\item Strangle for each delta.
\end{enumerate}
The strikes/volatilities table is obtained through
\begin{enumerate}
\item Compute the wing volatilities from the risk reversals and strangles.
\item Compute the ATM strike from the ATM volatility.
\item Compute the put/call strikes for each delta from the wing volatilities with (\ref{EqnDelta}).
\end{enumerate}

In the current implementation, the volatility is interpolated linearly on strikes and extrapolated flat beyond the extreme strikes.

\TODO{Add a smooth interpolation approach.}

\section{Currency exposure}

For non-forex instruments and forex forward transactions, the currency risk is simply the present value in each currency. 

This is obviously not the case for currency options. For currency options it is the amount in each currency required to neutralise the currency movements. The neutrality will depend on the model used for the option valuation. Here we use the Black model (with the volatility provided by the smile description). By using the Black model, the change of volatility implied by the change of spot is not taken into account.

For vanilla options the currency exposure is computed in the following way: for the foreign currency the exposure is
\[
\Delta_\spot N_1
\]
and for the domestic currency it is
\[
-\Delta_\spot N_1 \spot + \pv.
\]

In other world, if one buys an option at the fair price and at the same time sells $\Delta_\spot N_1$ foreign currency unit against receiving $\Delta_\spot N_1 \spot$ domestic currency unit, he will have no currency position. The $\pv$ part of the currency exposure hedges the premium paid in the domestic currency.

\section{Implementation}

The class implementing the computation of the implied strike from a delta and a volatility is
\class{BlackImpliedStrikeFromDeltaFormula}.

The smile data for one time to expiry can be stored in the class
\class{SmileDeltaParameter}. It can be constructed from deltas, ATM, strangles and risk reversals as described above.

The term structure version of the data consisting of an array of the previous parameters and an array of times to expiry can be stored in the class
\class{SmileDeltaTermStructureParameter}.

The pricing method for vanilla Forex option is \class{ForexOptionVanillaMethod}. There is a present value method and a currency exposure method.

\bibliography{../bibtex/finance}
\bibliographystyle{apa}

\tableofcontents

\end{document}

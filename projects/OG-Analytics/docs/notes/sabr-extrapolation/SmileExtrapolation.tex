\documentclass[]{amsart}

\usepackage{amsmath}
\usepackage{amssymb}
\usepackage{color}
\usepackage[
   pagebackref=true,
   colorlinks=true,
   urlcolor=ogblue,
   citecolor=oggreen,
   linkcolor=ogblue]{hyperref}
   \usepackage{natbib}
\usepackage{pgf}

\pgfdeclareimage[height=0.7cm]{oglogo}{../logo/OpenGammaLogo}

\definecolor{ogblue}{RGB}{0,85,129}
\definecolor{oggreen}{RGB}{93,194,164}
\definecolor{ogbluelight}{RGB}{17,116,158} 
\definecolor{ogred}{RGB}{204,44,0}
\definecolor{oggray}{RGB}{178,178,178} 
\definecolor{oggraylight}{RGB}{234,234,234}

\setlength{\oddsidemargin}{0.5cm}
\setlength{\evensidemargin}{0.5cm}
\setlength{\textwidth}{15cm}
\setlength{\textheight}{22cm}

\newcommand{\TODO}[1]%
   {{\sffamily \textbf{\color{oggreen}\noindent TO DO}\marginpar{\hspace*{0.5cm}
     \sffamily \textbf{\color{oggreen}TO DO}}: #1}}
     
\newcommand{\black}{{\operatorname{Black}}}
\newcommand{\pv}{{\operatorname{PV}}}
\newcommand{\class}[1]{{\texttt{#1}}}
     
\title[Smile extrapolation]%
   {Smile extrapolation}

\author[OpenGamma]%
   {OpenGamma Quantitative Research}

\date{First version: 21 April 2011; this version: 6 May 2011}

\thanks{Version 1.2}

\address{Marc Henrard \\ Quantitative Research \\ OpenGamma}

\email{quant@opengamma.com}

\begin{document}

\maketitle

\begin{center}
\pgfuseimage{oglogo} 
\end{center}

\begin{abstract}
An implementation of smile extrapolation for high strikes is described. The main smile is described by a implied volatility function, SABR for example. The extrapolation described is available for cap/floor and swaption pricing in OpenGamma library.
\end{abstract}

\section{Introduction}

The \emph{smile} is the description of the strike dependency of option price through Black implied volatilities. The name come from the shape of the curve which for most markets resemble a smile.

For interest rate markets like cap/floor and swaption, the smile is often described using a model like SABR. More exactly an approximated implied volatility function is used to obtained the prices. The model parameters are calibrated to fit the smile for the quoted options, which usually have a strike in the at-the-money region.

The standard approximated methods do not produce very good results far away from the money. Moreover the model calibration close to the money does not necessarily provide relevant information for far away strikes. When only vanilla options are priced, these problems are relatively minor as the time value of those far away from the money options is small. 

This is not the case for some instruments the pricing of which depends on the full smile. In the cap/floor world, those instruments include the swap and cap/floor in arrear and the swaps with long or short tenors. In the swaption world, they include the CMS swap and cap/floor (pre and post-fixed).

The requirements for the extrapolation, above the easiness to implement, is that they should provide arbitrage free extrapolation and leave a degree of freedom on the tail to calibrate to the traded smile dependent products. The degree of freedom will be refer to as the \emph{tail control} or \emph{tail thickness}.

The implementation described here is based on \cite{BDK.2008.1}. The call prices are extrapolated; the put prices are obtained by put/call parity.

\section{Notation}

The analysis framework is the pricing of option through a Black formula and an implied volatility description. In the main region, the call prices are obtained by the formula
\begin{equation}
\label{EqnPrice}
P(F,K) = N \black(F, K, \sigma(F,K)).
\end{equation}
where $F$ is the forward, $K$ the strike, $N$ the numeraire and $\sigma$ the implied Black volatility.

\section{Extrapolation}

A cut-off strike $K^*$ is selected. Below that strike the pricing formula described above is used. Above that strike an extrapolation of call prices on strikes is used. The shape of the extrapolation is based on prices (and not on volatilities). The functional form of the extrapolation is taken from \cite{BDK.2008.1} and is given by
\begin{equation}
\label{EqnExtra}
f(K) = K^{-\mu} \exp\left( a + \frac{b}{K} + \frac{c}{K^2} \right).
\end{equation}
The parameter $\mu$ will be used to control the tail. The other three parameters are used to ensure regularity.
\subsection{Gluing}

The \emph{gluing} between the two parts requires some regularity condition. It is required\footnote{Actually some $C^1$ condition would be sufficient but could produce a jump in the cumulative density and an atomic weight on the density.} to be $C^2$. 

To be able to achieve this smooth gluing, we need to compute the derivatives of the price with respect to the strike up to the second order. The derivatives of the price are, for the first order,
\[
D_K P(F,K) = N \left( D_K \black(F, K, \sigma(F,K)) + D_\sigma \black(F, K, \sigma(F,K) D_K \sigma(F,K) \right)
\]
and, for the second order,
\begin{eqnarray*}
D^2_{K,K} P(F,K) & = & N \left(\vphantom{\int} D^2_{K,K} \black(F, K, \sigma) +D^2_{\sigma,K} \black(F, K, \sigma) D_K \sigma(F,K) \right.\\
& & \quad + \left( D^2_{K,\sigma}\black(F, K, \sigma) + D^2_{\sigma,\sigma}\black(F, K, \sigma) D_K \sigma(F,K) \right) D_K \sigma(F,K) \\
& & \quad \left. +  D_\sigma \black(F, K, \sigma(F,K)) D^2_{K,K} \sigma(F,K) \vphantom{\int}\right).
\end{eqnarray*}

\subsection{Extrapolation derivatives}
The functional for the extrapolation is
\[
f(K) = K^{-\mu} \exp\left( a + \frac{b}{K} + \frac{c}{K^2} \right).
\]
Its first and second order derivatives are required to ensure the smooth gluing. The derivatives are
\[
f'(K) = K^{-\mu} \exp\left( a + \frac{b}{K} + \frac{c}{K^2} \right) (-\mu K^{-1} - bK^{-2}-2cK^{-3}) = f(K)(-\mu K^{-1} - bK^{-2}-2cK^{-3})
\]
and
\begin{eqnarray*}
f''(K) & = & f(K) \left(\vphantom{\int} \mu(\mu+1)K^{-2}+2b(\mu+1)K^{-3}+2c(2\mu+3)K^{-4} + b^2 K^{-4} + 4bc K^{-5} + 4 c^2K^{-6} \right).
\end{eqnarray*}

\subsection{Fitting}

The two parts are fitted at the cut-off strike. At that strike the price and its derivatives are $P(K^*,F) = p$, $\partial P/\partial K(K^*,F) = p'$ and $\partial^2 P/\partial K^2(K^*,F) = p''$.

To obtain the extrapolation parameters ($a$, $b$, $c$), a system of three equation with three unknowns has to be solved. Due to the structure of the function, it is possible to isolate the first variable ($a$) and to solve a simpler system with two equations with two unknowns:
\[
p' = p\left(-\mu K^{-1} - bK^{-2}-2cK^{-3}\right)
\]
\[
p'' = p \left(\vphantom{\int} \mu(\mu+1)K^{-2}+2b(\mu+1)K^{-3}+2c(2\mu+3)K^{-4} + b^2 K^{-4} + 4bc K^{-5} + 4 c^2K^{-6} \right).
\]
The parameter $a$ can be written explicitly in term of $b$ and $c$:
\[
a = \ln(p K^{\mu}) - \frac{b}{K} - \frac{c}{K^2}.
\]

\section{Examples}

Some examples of smile extrapolation are presented. The SABR data used is $\alpha=0.05$, $\beta=0.50$, $\rho=-0.25$ and $\nu=0.50$. The forward is 5\%, the cut-off strike 10\% and the time to expiry 2. 

The extrapolation is computed for $\mu=5$, 40, 90 and 150. The choices of the exponents have been exaggerated to obtain clearer pictures.

%TODO: produce better pictures.

In Figure~\ref{FigPrice} the tail of the price (not adjusted by the numeraire) is given.

\pgfdeclareimage[height=5.0cm]{extraPrice}{ExtrapolationPrice}
\begin{figure}
\begin{center}
\pgfuseimage{extraPrice}
\end{center}
\caption{Price extrapolation}
\label{FigPrice}
\end{figure}

In Figure~\ref{FigDensity} the tail of the price density is given.

\pgfdeclareimage[height=5.0cm]{extraDensity}{ExtrapolationDensity}
\begin{figure}
\begin{center}
\pgfuseimage{extraDensity}
\end{center}
\caption{Price extrapolation: impact on density}
\label{FigDensity}
\end{figure}

In Figure~\ref{FigSmile} the tail of the smile is given.

\pgfdeclareimage[height=5.0cm]{extraSmile}{ExtrapolationSmile}
\begin{figure}
\begin{center}
\pgfuseimage{extraSmile}
\end{center}
\caption{Price extrapolation: impact on volatility smile}
\label{FigSmile}
\end{figure}

The impact of the smile extrapolation on the CMS prices is important. In Table~\ref{TabCMS}, a pre-fixed CMS coupon with around 9Y on a 5Y swap is computed. Depending on the tail, the CMS convexity adjustment can be very important (up to 50bps difference). The values of $mu$ have been selected to have differences of around 10 bps in adjusted rate between them. 

\begin{table}
\begin{tabular}{lrr}
Method & Parameter ($\mu$) & Adjusted rate (in \%) \\
\hline
SABR & -- & 5.08\\
SABR with extrapolation & 1.30 & 5.03\\
SABR with extrapolation & 1.55 & 4.98\\
SABR with extrapolation & 2.25 & 4.88\\
SABR with extrapolation & 3.50 & 4.78\\
SABR with extrapolation & 6.00 & 4.68\\
SABR with extrapolation & 15.00 & 4.58\\
No convexity adjustment & -- & 4.04
\end{tabular}
\caption{CMS prices with different methods.}
\label{TabCMS}
\end{table}

\section{Derivatives}

To compute the prices derivatives (greeks) with respect to the different market input, it is useful to have the derivatives of the building blocks. In this case, we would like to have the derivative of the price $f(K)$ with respect to the forward $F$ and the parameters describing the volatility surface $\sigma$ (called $p$ hereafter). 

From the way the price is written it may appear that $f(K)$ does not depend of $F$ and $p$. Actually the dependency is hidden into the parameters $a$, $b$ and $c$. To see that we rewrite the problem that was solved. Let $\tilde f$ denote the vector function $(f, f', f'')$ and $\tilde P$ the vector function $(P, \partial P / \partial K, \partial^2 P/\partial K^2)$. The equation solved to obtain $x$ is
\[
\tilde f(K^*, x) = \tilde P (F, K^*, p).
\]

For the sequel the cut-off strike $K^*$ is a constant and we ignore it to simplify the writing. The equation to solve is then
\[
g(x, F, p) = \tilde f(x) - \tilde P(F, p)=0.
\]
The equation is usually solved by numerical techniques and we have only one solution point but we would like to compute the derivative of the solution around the initial point. We have, at the initial point $(F_0, p_0)$,
\[
x_0 = x(F_0, p_0)
\]
which solves
\[
g(x_0, F_0, p_0) = 0
\]
and we look for 
\[
D_F x(F_0, p_0) \quad \mbox{and} \quad D_p x(F_0, p_0).
\]
Unfortunately $x$ is not known explicitly, we only know that such a function $x(F,p)$ should exists around $(F_0, p_0)$. To get the required derivative we rely on the \emph{implicit function theorem} (add a reference) which stated the existence of such a function and gives its derivative from $g$ derivatives. If $D_x g(x_0, F_0, p_0)$ is invertible, then the implicit function $x(F,p)$ that solve the equation
\[
g(x(F,p), F, p)=0
\]
exists around $(F_0,p_0)$ and
\[
D_F x(F_0, p_0) = - (D_x g(x_0, F_0, p_0))^{-1}D_F g(x_0, F_0, p_0)
\]
and 
\[
D_p x(F_0, p_0) = - (D_x g(x_0, F_0, p_0))^{-1}D_p g(x_0, F_0, p_0).
\]


\section{Implementation}

In the OpenGamma library the extrapolation is implemented in the class \class{SABRExtrapolationRightFunction}. The call prices are extrapolated and the put prices are obtained by put/call parity.

The extrapolation is used for the swaption pricing in \class{SwaptionPhysicalFixedIborSABRExtrapolationRightMethod} and the CMS pricing in the class \class{CouponCMSSABRExtrapolationRightReplicationMethod}.

To implement that formula, the following derivatives should be available:
\[
D_K \black, D_\sigma \black, D^2_{K,K} \black, D^2_{\sigma,K}\black, D^2_{\sigma,\sigma}\black
\]
\[
D_K \sigma, D^2_{K,K} \sigma
\]

The required partial derivatives of the Black formula and of the \cite{HKL.2002.1} SABR functional formula are also available in the library. The Black formula is available in the class \class{BlackPriceFunction} and the partial derivatives of first and second order are available in the method \class{getPriceAdjoint2}. The SABR approximated formula is available in the class \class{SABRHaganVolatilityFunction} and the required first and second order derivatives are in the method \class{getVolatilityAdjoint2}.

For the Black formula, the computation time for the three first order derivatives (forward, strike, volatility) and three second order derivatives (strike-strike, strike-volatility, volatility-volatility) is around 1.75 times the price time. A finite difference implementation would require around 7 times the time of a price and provide less stability.

Obviously the implementation with extrapolation will be slower than the implementation without extrapolation. For swaptions, the SABR with extrapolation takes around twice the time for the standard SABR in the extrapolated region.

\bibliography{../bibtex/finance}
\bibliographystyle{apa}

\tableofcontents

\end{document}